\documentclass{article}
\usepackage{amsmath}
\usepackage{amssymb}
\usepackage{geometry}
\geometry{margin=1in}


\section*{Problem Statement}
Let $x$ be a string of length $n$ and $s$ be a string of length $m$, where both strings consist of lowercase Latin letters. You are allowed to repeatedly append the current string $x$ to itself (i.e., $x \leftarrow x + x$). After each operation, the length of $x$ doubles.

We want to determine the minimum number of operations after which $s$ appears as a substring of $x$, or conclude that this is impossible. It is given that:
\[
n \cdot m \leq 25
\]

\section*{Goal}
We aim to prove the following:

\textit{If $s$ can appear in $x$ as a substring after some number of self-appending operations, then it will necessarily appear within 5 operations (i.e., when $\lvert x \rvert = 32n$).}

\section*{Operation Growth}
Each operation doubles the length of $x$. After $k$ operations, the length of $x$ becomes:
\[
|x_k| = n \cdot 2^k
\]

Thus, the growth of $x$ over the first few operations is:

\begin{center}
\begin{tabular}{|c|c|}
\hline
Operations & Length of $x$ \\
\hline
0 & $n$ \\
1 & $2n$ \\
2 & $4n$ \\
3 & $8n$ \\
4 & $16n$ \\
5 & $32n$ \\
\hline
\end{tabular}
\end{center}

\section*{Key Insight: Repetition Structure}
Let $x_0$ be the original string. Each subsequent $x_k$ is a repetition of $x_0$:
\[
x_k = \underbrace{x_0 \; x_0 \; \dots \; x_0}_{2^k \text{ times}}
\]

Therefore, all versions of $x$ are made solely by concatenating copies of the original $x_0$, and any substring that appears in $x_k$ must consist of parts (or overlaps) of these repeated $x_0$ segments.

\section*{What Must Be Guaranteed}
To ensure that $s$ appears as a substring of $x_k$:
\begin{enumerate}
    \item $x_k$ must be at least as long as $s$ (i.e., $|x_k| \geq m$).
    \item $x_k$ must be long enough to allow all possible \textbf{alignments} of $s$ to be tried — especially those where $s$ spans across boundaries between adjacent repetitions of $x_0$.
\end{enumerate}

\section*{Worst-Case Overlap Alignment}
Suppose $s$ starts near the end of one repetition of $x_0$ and ends at the beginning of the next. In such a case, $s$ overlaps multiple repetitions of $x_0$. To handle this, $x_k$ must be long enough to simulate a sliding window of size $m$ over all possible positions.

If $k$ is the number of repetitions of $x_0$, then the total length of $x$ is $k \cdot n$. The number of $m$-length substrings available in such a string is:
\[
k \cdot n - m + 1
\]

To ensure that $s$ is \textbf{guaranteed} to appear (if possible), a sufficient condition is:
\[
k \cdot n \geq m + n - 1
\]

Even more conservatively, we can demand:
\[
k \cdot n \geq 2m
\]
which ensures enough overlap between repetitions to cover all alignment positions.

\section*{Using the Problem Constraint}

We are given:
\[
n \cdot m \leq 25
\Rightarrow m \leq \frac{25}{n}
\]

We require:
\[
32n \geq 2m \Rightarrow m \leq 16n
\]

This always holds because:
\[
\frac{25}{n} \leq 16n \quad \text{for all } n \in [1, 25]
\]

Proof of this inequality:

Multiply both sides:
\[
25 \leq 16n^2 \Rightarrow n^2 \geq \frac{25}{16} = 1.5625 \Rightarrow n \geq \sqrt{1.5625} \approx 1.25
\]

Since $n$ is a positive integer and $n \geq 1$, the inequality always holds.

\section*{Conclusion}

After 5 operations, the string $x$ has length $32n$, which satisfies:
\begin{itemize}
    \item $32n \geq 2m$
    \item Covers all possible alignments of $s$ of length $m$
    \item Adheres to the global constraint $n \cdot m \leq 25$
\end{itemize}

Thus, if $s$ does not appear as a substring of $x$ after 5 operations, it will never appear in any longer repetition of $x$.

\[
\boxed{
    \text{If } s \text{ can appear in } x, \text{ it will appear within 5 operations (i.e., length } 32n\text{).}
}
\]

\end{document}
