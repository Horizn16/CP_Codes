\documentclass{article}
\usepackage{amsmath}
\usepackage{amssymb}
\usepackage{enumitem}
\usepackage{geometry}
\geometry{margin=1in}

\title{XOR of \( a_i \oplus x \) and Behavior When XORed Together}
\date{}
\begin{document}

\maketitle

\section*{Step 1: Write the Expression Clearly}

\[
\bigoplus_{i=1}^{n} (a_i \oplus x)
\]

This means:

\[
(a_1 \oplus x) \oplus (a_2 \oplus x) \oplus \cdots \oplus (a_n \oplus x)
\]

\section*{Step 2: Use XOR Properties (Associative and Commutative)}

Because XOR is associative and commutative, we can rearrange terms freely:

\[
\bigoplus_{i=1}^{n} (a_i \oplus x) = \left( \bigoplus_{i=1}^{n} a_i \right) \oplus \left( \bigoplus_{i=1}^{n} x \right)
\]

\section*{Step 3: Simplify \( \bigoplus_{i=1}^{n} x \)}

Since \( x \) is XORed with itself \( n \) times:

\begin{itemize}
  \item If \( n \) is even, then: \( x \oplus x \oplus \cdots \oplus x = 0 \)
  \item If \( n \) is odd, then: \( x \oplus x \oplus \cdots \oplus x = x \)
\end{itemize}

\subsection*{Final formula:}

\[
\bigoplus_{i=1}^{n} (a_i \oplus x) = 
\left( \bigoplus_{i=1}^{n} a_i \right) \oplus 
\begin{cases}
0 & \text{if } n \text{ even} \\
x & \text{if } n \text{ odd}
\end{cases}
\]

\section*{Intuition}

XORing all the \( a_i \) first, then XORing with \( x \) if the count \( n \) is odd.  
If \( n \) is even, all the \( x \)'s cancel out.

\section*{How Does This Combination Happen?}

Starting with:

\[
\bigoplus_{i=1}^{n} (a_i \oplus x) = (a_1 \oplus x) \oplus (a_2 \oplus x) \oplus \cdots \oplus (a_n \oplus x)
\]

\textbf{Step 1: Expand XORs inside parentheses:}

\[
= a_1 \oplus x \oplus a_2 \oplus x \oplus \cdots \oplus a_n \oplus x
\]

\textbf{Step 2: Use commutativity to rearrange:}

\[
= a_1 \oplus a_2 \oplus \cdots \oplus a_n \oplus x \oplus x \oplus \cdots \oplus x
\]

\textbf{Step 3: Use associativity to regroup:}

\[
= \left( \bigoplus_{i=1}^{n} a_i \right) \oplus \left( \bigoplus_{i=1}^{n} x \right)
\]

\textbf{Step 4: Simplify the \( x \)'s:}

\[
x \oplus x = 0
\]

So, depending on \( n \):

\[
\bigoplus_{i=1}^{n} x = 
\begin{cases}
0 & \text{if } n \text{ even} \\
x & \text{if } n \text{ odd}
\end{cases}
\]

\subsection*{Summary:}

\[
\bigoplus_{i=1}^{n} (a_i \oplus x) = 
\left( \bigoplus_{i=1}^{n} a_i \right) \oplus 
\begin{cases}
0 & \text{if } n \text{ even} \\
x & \text{if } n \text{ odd}
\end{cases}
\]

\section*{Why Does This Happen?}

Because XOR is like addition modulo 2, and it cancels pairs of identical elements.  
So repeated \( x \)'s pair up and vanish if even in count, but if odd, one \( x \) remains.

\section*{Quick Example}

Let \( n = 3 \), \( a_1 = 5 \), \( a_2 = 7 \), \( a_3 = 2 \), \( x = 4 \)

Compute:

\[
(5 \oplus 4) \oplus (7 \oplus 4) \oplus (2 \oplus 4)
\]

Step-by-step:

\begin{align*}
5 \oplus 4 &= 1 \\
7 \oplus 4 &= 3 \\
2 \oplus 4 &= 6 \\
1 \oplus 3 \oplus 6 &= (1 \oplus 3) \oplus 6 = 2 \oplus 6 = 4
\end{align*}

Now compute XOR of \( a_i \)'s:

\[
5 \oplus 7 \oplus 2 = (5 \oplus 7) \oplus 2 = 2 \oplus 2 = 0
\]

Since \( n = 3 \) is odd:

\[
0 \oplus 4 = 4
\]

Matches the earlier result!

\section*{Why Can You Remove Brackets in XOR Expressions?}

Because XOR is:

\begin{itemize}
  \item \textbf{Associative:} \( (a \oplus b) \oplus c = a \oplus (b \oplus c) \)
  \item \textbf{Commutative:} \( a \oplus b = b \oplus a \)
\end{itemize}

\subsection*{What Does This Imply?}

With associativity, you can remove brackets safely and XOR everything in a single chain:

\[
(a_1 \oplus a_2) \oplus a_3 = a_1 \oplus a_2 \oplus a_3
\]

With commutativity, you can reorder terms however you like without changing the result.

\section*{Removing Brackets Step-by-Step}

Starting with:

\[
((a_1 \oplus x) \oplus (a_2 \oplus x)) \oplus (a_3 \oplus x)
\]

Remove brackets:

\[
= a_1 \oplus x \oplus a_2 \oplus x
\Rightarrow a_1 \oplus a_2 \oplus x \oplus x
= a_1 \oplus a_2 \oplus 0 = a_1 \oplus a_2
\]

Now XOR with \( a_3 \oplus x \):

\[
(a_1 \oplus a_2) \oplus a_3 \oplus x = a_1 \oplus a_2 \oplus a_3 \oplus x
\]

\subsection*{Summary}

Removing brackets in XOR is safe because XOR is associative.  
You can rearrange and regroup terms freely.  
No ambiguity or issues arise from dropping parentheses in XOR expressions.

\section*{Conclusion}

\[
\bigoplus_{i=1}^{n}(a_i \oplus x) =
\left(\bigoplus_{i=1}^{n} a_i \right) \oplus 
\begin{cases}
0 & \text{if } n \text{ is even} \\
x & \text{if } n \text{ is odd}
\end{cases}
\]

This identity is valid due to XOR’s properties: associativity, commutativity, and self-cancellation.

\end{document}
