\documentclass[12pt]{article}
\usepackage{amsmath}
\usepackage{amssymb}

\title{Achievability of All Intermediate Sums of $k$-Element Subsets from $\{1, 2, \dots, n\}$}
\author{}
\date{}

\begin{document}

\maketitle

It is clear that the minimum sum is obtained for the numbers $1,2,3,\dots,k$, and its value is 
$$\frac{k(k+1)}{2}$$
(the sum of the first $k$ natural numbers).

Furthermore, it is evident that the maximum sum is achieved for the numbers $n, n-1, n-2, \dots, n-k+1$, and its value is 
$$\frac{n(n+1)}{2} - \frac{(n-k)(n-k+1)}{2}$$
(the sum of all numbers from $1$ to $n$ minus the sum of all numbers from $1$ to $n-k$).

Let's prove that among any $k$ numbers (whose sum is not maximal), there exists a number $a < n$ such that $a+1$ is not among those $k$ numbers.

Let's assume the opposite, that is, there exist $k$ numbers whose sum is not maximal, and for each of those $k$ numbers, $a+1$ is also among those numbers. Let $v$ be the smallest among them. Consequently, $v+1$ is also among these $k$ numbers. Since $v+1$ is in these $k$ numbers, then $v+2$ is also among these $k$ numbers. Similarly, we can conclude that 
$$v, v+1, v+2, v+3, \dots$$
are all among these $k$ numbers. However, since we have $k$ of them, these are the $k$ numbers that would yield the maximum sum 
$$n, n-1, n-2, \dots, n-k+1.$$
This is a contradiction!

So, among any $k$ numbers (whose sum is not maximal), there exists a number $a < n$ such that $a+1$ is not among those $k$ numbers. Based on this, starting from the minimum sum $S$, we can obtain $S+1$ (by replacing the number $a$ with $a+1$, the sum increases by $1$), then from the sum $S+1$, we obtain the sum $S+2$, and so on.

Therefore, by applying the principle of mathematical induction, we can obtain any sum that is greater than or equal to the minimum sum and less than or equal to the maximum sum.

\bigskip

\textbf{Explanation of the Proof:}

We want to show that:

\textit{For any integer between the minimum and maximum possible sums of a subset of $k$ numbers from $\{1, 2, \dots, n\}$, that integer can be realized as the sum of some $k$-element subset.}

\bigskip

\textbf{Problem Setting:}

You are selecting $k$ numbers from the set $\{1, 2, \dots, n\}$.

\begin{itemize}
    \item The minimum sum is:
    $$1 + 2 + \dots + k = \frac{k(k+1)}{2}$$
    
    \item The maximum sum is:
    $$n + (n-1) + \dots + (n-k+1) = \frac{n(n+1)}{2} - \frac{(n-k)(n-k+1)}{2}$$
\end{itemize}

\bigskip

\textbf{Goal:} We want to prove that every integer between the minimum and maximum sums can be achieved by some $k$-element subset.

\bigskip

\textbf{Strategy:} We use mathematical induction and a contradiction-based exchange argument.

\bigskip

\textbf{Step 1: Base Case}

The minimum sum $S_{\text{min}} = \frac{k(k+1)}{2}$ is clearly achieved by choosing $\{1, 2, \dots, k\}$.

\bigskip

\textbf{Step 2: Inductive Step}

Suppose we have a subset whose sum is $S$ and we want to construct another subset whose sum is $S+1$, provided $S + 1 \leq S_{\text{max}}$.

If we can do this, we can continue the process until we reach the maximum.

\bigskip

\textbf{Step 3: Exchange Argument (Contradiction)}

Assume we have a set of $k$ elements whose sum is not maximal, but we cannot replace any element $a$ with $a+1$.

This implies that for every $a$ in our subset, $a+1$ is also in the subset.

Let $v$ be the smallest element in the subset. Then:
$$v \in \text{subset} \Rightarrow v+1 \in \text{subset} \Rightarrow v+2 \in \text{subset} \Rightarrow \dots \Rightarrow v + k - 1 \in \text{subset}$$

Hence, the subset is:
$$\{v, v+1, v+2, \dots, v+k-1\}$$

This is a block of $k$ consecutive numbers. For the sum to be maximum, this block must be:
$$\{n-k+1, n-k+2, \dots, n\}$$

So if $v + k - 1 = n$, we have the maximum sum — contradicting our assumption.

Therefore, our assumption is false. That is, there exists some $a < n$ such that $a + 1$ is not in the subset.

This means we can increase the sum by replacing $a$ with $a + 1$.

\bigskip

\textbf{Step 4: Applying Induction}

From the minimum sum, we can step-by-step increase the sum by replacing such $a$ values, until we reach the maximum.

Thus, every sum from the minimum to the maximum can be achieved.

\bigskip

\textbf{Final Conclusion:}

By the principle of mathematical induction and using the exchange argument, we conclude that every integer sum between:
$$\frac{k(k+1)}{2} \quad \text{and} \quad \frac{n(n+1)}{2} - \frac{(n-k)(n-k+1)}{2}$$
is achievable as the sum of some $k$-element subset of $\{1, 2, \dots, n\}$.

\end{document}
